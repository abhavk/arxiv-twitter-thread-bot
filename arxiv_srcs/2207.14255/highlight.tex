%https://tex.stackexchange.com/questions/15237/highlight-text-in-code-listing-while-also-keeping-syntax-highlighting/15257#15257
\usepackage{listings}
\usepackage{xcolor}
\usepackage{atbegshi}
\usepackage{ifthen}

\usepackage{tikz}
\usetikzlibrary{calc}
\usetikzlibrary{fit}

\makeatletter

\newcommand\if@empty[1]{%
    \if\relax\detokenize{#1}\relax
        \expandafter\@firstoftwo
    \else
        \expandafter\@secondoftwo
    \fi
}

\newcommand\ifthen[2]{\ifthenelse{#1}{#2}{}}
\newcommand\ifelse[2]{\ifthenelse{#1}{}{#2}}

% Is highlighting currently active in the listings environment?
\newif\ifhl@active

% Material executed at each page shipout
\newtoks\hl@shipout

% Extra styling parameters for the currently active highlight
\newtoks\hl@parameters

% Inline highlight style. This does not break across line ends
\newcommand\hl[2][]{\begingroup\hlstyle@start{#1}#2\endgroup}

% The highlight style for listings environments
\newcommand\hlstyle[1][]{\hlstyle@start{#1}}
\newcommand\hlstyle@start[1]{%
    \global\hl@activetrue
    \aftergroup\hlstyle@end
    \write\@auxout{\noexpand\hl@setparameters{#1}}%
    \hl@mark{S}%
}
\newcommand\hlstyle@end{%
    \hl@mark{E}%
    \global\hl@activefalse
}

% Sets the current highlight parameters
\newcommand\hl@setparameters[1]{%
    \global\hl@parameters={#1}%
}

% Writes a new node marker with the current position and page number to the .aux file
\newcommand\hl@mark[1]{%
    \ensuremath{\vcenter{\hbox{\pdfsavepos}}}%
    \write\@auxout{\noexpand\hl@processmark{#1}{\the\pdflastxpos}{\the\pdflastypos}{\arabic{page}}}%
}

% Writes the current page origin to the .aux file, and issues the current \hl@shipout material
% to be processed
\AtBeginShipout{\AtBeginShipoutUpperLeft{%
    \pdfsavepos
    \write\@auxout{\noexpand\hl@processmark{Z}{\the\pdflastxpos}{\the\pdflastypos}{\arabic{page}}}%
    \hl@doshipout
}}

% Maintains the current first and last node positions, checks whether a node marker should be used
% for the current highlighting segment, and issues a \hl@draw call if necessary.
% #1 is either 'S' (start node), 'M' (intermediate node), 'E' (end node), or 'Z' (page origin)
% #2/#3 are the x/y-coordinates, #4 is the page number
\newcommand\hl@processmark[4]{%
    \if#1S%
        \xdef\hl@firstx{#2}%
        \xdef\hl@firsty{#3}%
        \xdef\hl@firstp{#4}%
    \else\if#1M%
        \ifthenelse{\hl@firsty=#3}{%
            % Still on same line
            \xdef\hl@lastx{#2}%
            \xdef\hl@lasty{#3}%
        }{%
            \ifelse{\hl@firstx=\hl@lastx \and \hl@firsty=\hl@lasty}{%
                % Last highlight part is not empty
                \expandafter\hl@draw\expandafter{\the\hl@parameters}%
            }%
            \xdef\hl@firstx{#2}%
            \xdef\hl@firsty{#3}%
            \xdef\hl@firstp{#4}%
        }%
    \else\if#1E%
        \ifthen{\hl@firsty=#3}{%
            % Still on same line
            \xdef\hl@lastx{#2}%
            \xdef\hl@lasty{#3}%
            \ifelse{\hl@firstx=#2}{%
                % Last highlight part is not empty
                \expandafter\hl@draw\expandafter{\the\hl@parameters}%
            }%
        }%
    \else\if#1Z%
        \xdef\hl@zerox{#2}%
        \xdef\hl@zeroy{#3}%
    \fi\fi\fi\fi
}

% Add a new highlight segment to \hl@shipout to be processed on every shipout.
% #1 is the styling command used to draw the segment within a properly positioned \tikz call.
%    If #1 is empty, \hldefaultstyle is used as a default. Every styling command can take
%    arbitrary parameters, only the last four are mandatory (x_first, y_first, x_last, y_last)
\newcommand\hl@draw[1]{%
    \edef\@temp{{\noexpand\tikz [ overlay, shift = {(-\hl@zerox sp, -\hl@zeroy sp)} ]
        {\if@empty{#1}{\noexpand\hldefaultstyle}{\unexpanded{#1}}%
            {\hl@firstx sp}{\hl@firsty sp}{\hl@lastx sp}{\hl@lasty sp}}}%
        {\hl@firstp}%
    }%
    \expandafter\global\expandafter\addto@hook\expandafter\hl@shipout\expandafter{\@temp}%
}

% Processes the current \hl@shipout material. The material is a sequence of
% {TikZ commands}{page number} pairs
\def\hl@doshipout{%
    \expandafter\hl@doshipout@\the\hl@shipout{}{}\@end
}
\def\hl@doshipout@#1#2#3\@end{%
    \if@empty{#2}{}{%
        \ifthen{\arabic{page}=#2}{#1}%
        \hl@doshipout@#3\@end
    }%
}

% Hook into listings' discretionary breaks
\let\orig@lst@discretionary=\lst@discretionary
\gdef\lst@discretionary{%
    \ifhl@active
        \hl@mark{M}%
    \fi
    \orig@lst@discretionary
    \ifhl@active
        \hl@mark{M}%
    \fi
}


% Styling command for underlined text
\newcommand\hlunderlinestyle[5][black]{%
    \draw [ yshift = -0.4\baselineskip, #1 ] (#2, #3) -- (#4, #5);
}

% Styling command for a simple marker highlights
\newcommand\hlmarkerstyle[5][yellow]{%
    \draw [ line width = \baselineskip, #1 ] (#2, #3) -- (#4, #5);
}

% Styling command for a framed box
\newcommand\hlboxstyle[5][red]{%
    \draw let \p1 = (#2, #3), \p2 = (#4, #5)
          in node [ draw = #1,
                    fill = #1!10!white,
                    minimum height = \baselineskip,
                    fit = (\p1)(\p2),
                    anchor = west,
                    align = none,  % prevent underfull node boxes
                    outer sep = 0pt, inner sep = 0pt ]
             at (\p1) {};
}

% Default styling command
\let\hldefaultstyle=\hlmarkerstyle

\makeatother


\newcommand\delimstyleA{\hlstyle[{\hlboxstyle[green]}]}
\newcommand\delimstyleB{\hlstyle[{\hlboxstyle[green!20, draw=green]}]}
\newcommand\delimstyleBsm{\hlstyle[{\small\hlboxstyle[green!20, draw=green]}]}
\newcommand\delimstyleBft{\hlstyle[{\footnotesize\hlboxstyle[green!20, draw=green]}]}
\newcommand\delimstyleC{\hlstyle[{\hlunderlinestyle[red, line width = 0.8pt]}]}

\lstdefinestyle{python}
{
    %language=python,
    columns=fullflexible,
    basicstyle={\ttfamily \small},
    moredelim=**[is][\delimstyleA]{@}{@},
    moredelim=**[is][\delimstyleBsm]{`}{`},
    %moredelim=**[is][\delimstyleC]{_}{_},
    keepspaces,
    breaklines,
    breakautoindent=false,
    breakindent=0pt,
}

\lstdefinestyle{text}
{
    columns=fullflexible,
    basicstyle={\sffamily \footnotesize},
    moredelim=**[is][\delimstyleA]{@}{@},
    moredelim=**[is][\delimstyleBft]{`}{`},
    %leftmargin=0pt,
    %itemindent=3ex,
    %itemsep=-1ex,
    %moredelim=**[is][\delimstyleC]{_}{_},
    keepspaces,
    breaklines,
    breakautoindent=false,
    breakindent=0pt,
}